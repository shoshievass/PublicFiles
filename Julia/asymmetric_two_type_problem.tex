% Options for packages loaded elsewhere
\PassOptionsToPackage{unicode}{hyperref}
\PassOptionsToPackage{hyphens}{url}
%
\documentclass[
]{article}
\usepackage{lmodern}
\usepackage{amssymb,amsmath}
\usepackage{ifxetex,ifluatex}
\ifnum 0\ifxetex 1\fi\ifluatex 1\fi=0 % if pdftex
  \usepackage[T1]{fontenc}
  \usepackage[utf8]{inputenc}
  \usepackage{textcomp} % provide euro and other symbols
\else % if luatex or xetex
  \usepackage{unicode-math}
  \defaultfontfeatures{Scale=MatchLowercase}
  \defaultfontfeatures[\rmfamily]{Ligatures=TeX,Scale=1}
\fi
% Use upquote if available, for straight quotes in verbatim environments
\IfFileExists{upquote.sty}{\usepackage{upquote}}{}
\IfFileExists{microtype.sty}{% use microtype if available
  \usepackage[]{microtype}
  \UseMicrotypeSet[protrusion]{basicmath} % disable protrusion for tt fonts
}{}
\makeatletter
\@ifundefined{KOMAClassName}{% if non-KOMA class
  \IfFileExists{parskip.sty}{%
    \usepackage{parskip}
  }{% else
    \setlength{\parindent}{0pt}
    \setlength{\parskip}{6pt plus 2pt minus 1pt}}
}{% if KOMA class
  \KOMAoptions{parskip=half}}
\makeatother
\usepackage{xcolor}
\IfFileExists{xurl.sty}{\usepackage{xurl}}{} % add URL line breaks if available
\IfFileExists{bookmark.sty}{\usepackage{bookmark}}{\usepackage{hyperref}}
\hypersetup{
  pdftitle={Asymmetric (Two Type) Procurement Auction Equilibrium},
  pdfauthor={S. Vasserman},
  hidelinks,
  pdfcreator={LaTeX via pandoc}}
\urlstyle{same} % disable monospaced font for URLs
\usepackage[margin=1in]{geometry}
\usepackage{graphicx,grffile}
\makeatletter
\def\maxwidth{\ifdim\Gin@nat@width>\linewidth\linewidth\else\Gin@nat@width\fi}
\def\maxheight{\ifdim\Gin@nat@height>\textheight\textheight\else\Gin@nat@height\fi}
\makeatother
% Scale images if necessary, so that they will not overflow the page
% margins by default, and it is still possible to overwrite the defaults
% using explicit options in \includegraphics[width, height, ...]{}
\setkeys{Gin}{width=\maxwidth,height=\maxheight,keepaspectratio}
% Set default figure placement to htbp
\makeatletter
\def\fps@figure{htbp}
\makeatother
\setlength{\emergencystretch}{3em} % prevent overfull lines
\providecommand{\tightlist}{%
  \setlength{\itemsep}{0pt}\setlength{\parskip}{0pt}}
\setcounter{secnumdepth}{-\maxdimen} % remove section numbering

\title{Asymmetric (Two Type) Procurement Auction Equilibrium}
\author{S. Vasserman}
\date{6/19/2020}

\begin{document}
\maketitle

The goal of this note is extend the baseline model discussed in
\href{https://scholar.harvard.edu/files/vasserman/files/bv_sept2019.pdf}{Bolotnny
and Vasserman (2019)} to allow for two-dimensional types.

We begin by defining the primitives of the model in order of importance
for the exercise.

Bidders indexed by \(i\) are described types consisting of two
parameters: - A ``cost'' parameter: \(\alpha^i\) - This captures the
relative cost-efficiency of bidder \(i\). - A ``risk'' parameter:
\(\gamma^i\) - This captures the extent to which bidder \(i\) dislike
risks.

The role of \(\gamma\) is mediated by a ``utility'' function that models
how much a bidder values a given bet. In particular, we will use:
\[ u(\pi) = 1 - \exp(-\gamma \pi) \]

To solidify what this means, suppose that a bidder faces a bet: win
\$100 with probability 0.5 or \$0 with probability 0.5. If the bidder
has \(\gamma = 0.01\), then the expectation of his utility for the bet
is:
\[ 0.5 \cdot u(100) + 0.5 \cdot u(0) = 0.5 \cdot (1 - \exp(-\frac{100}{100})) + 0.5 \cdot (1 - \exp(-\frac{0}{100}))
\] \[ = 0.5 \cdot (1 - \exp(-1)) + 0.5 \cdot (0) \approx 0.316
\] By contrast, if he were able to get \$50 for sure, he would value
this at \[ u(50) = 1 - \exp(-\frac{50}{100}) \approx 0.393
\] so you can see that the bidder ``dislikes'' risk. In general, the
higher that \(\gamma\) is, the larger the discrepency between the value
of a ``bet'' and a ``sure thing'' of equal expected value.

The rest of the parameters in the model explain the way that a bidder of
type \((\alpha^i, \gamma^i)\) values a given bid that he is considering
to submit to the auction.

This is described in the problem definision section just below.

For each of \(t = 1, \ldots, T\) materials (items) that procurement
project will require, there is: - \(q_t^e\): - The DOT engineer's
estimate of the quantity of item \(t\) that will be needed for the
project - \(q_t^b\): - The bidder's estimated quantity of item \(t\)
that will be used - \(\sigma_t^2\) - The variance of the bidder's
estimate of \(q_t^b\)

\begin{itemize}
\tightlist
\item
  \(c_t\):

  \begin{itemize}
  \tightlist
  \item
    The market unit rate for item \(t\)
  \end{itemize}
\end{itemize}

Note that I often use ``DOT'' to refer to the government org that runs
the auction -- DOT stands for Department of Transportation.

\hypertarget{the-problem-definition}{%
\subsubsection{The Problem Definition}\label{the-problem-definition}}

We will consider an auction in which there are \(2\) kinds of bidders,
characterized by different values of \(\gamma\): \(\gamma_1\) and
\(\gamma_2\).

We are interested in solving for the ``equilibrium bidding'' function
that maps each possible type \((\tilde{\alpha}, \gamma_i)\) to a
``score'' \(s(\tilde{\alpha}, {\gamma}_i)\).

We can think the ``score'' function as a map: \[
s_{i} : [\underline{\alpha}, \overline{\alpha}] \times \gamma_i \rightarrow [\underline{S}, \overline{S}]
\]

where each of \([\underline{\alpha}, \overline{\alpha}]\) and
\([\underline{S}, \overline{S}]\) are subsets of \(\mathbb{R}_{+}\).

Note that each \(\gamma\)-type is going to have its own monotonic score
function. We have two \(\gamma\) types --- \(\gamma_1\) and \(\gamma_2\)
--- and so we will need to solve for two score functions ---
\(s_1(\alpha)\) and \(s_2(\alpha)\).

In order for a prospective ``score'' function to be an equilibrium, it
needs maximize the ``expected utility'' from participating in the
auction for each possible bidder type
\((\tilde{\alpha}, \tilde{\gamma})\) --- let's call this
\(EU(\tilde{\alpha}, \tilde{\gamma}, s(\tilde{\alpha}, \tilde{\gamma}))\).

This is the product of the (expected) utility that the bidder would get
from completing the project if he wins (given his bid) times the
probability that he wins.

\[
EU(\tilde{\alpha}, {\gamma_i}, s_{i}(\tilde{\alpha})) = \text{E}[u(\pi(\tilde{\alpha}, {\gamma_i}, s_{i}(\tilde{\alpha})) \text{ | win}] \times \text{Pr(win | } s_{i}(\tilde{\alpha}) \text{ )}
\]

Let's break this down. Suppose that bidder \(i\) drew a value
\(\alpha_i\) and bid according to the score function that we are trying
to define, knowing that bidder \(j\) drew some other (unown to \(i\))
value \(\alpha_j\) from the same IID distribution (e.g.~the uniform
distribution).

The rules of the auction are that the bidder with the \textbf{lowest}
score wins the auction. Thus, we can write 

\begin{align*}
\text{Pr(win | } s_i(\tilde{\alpha}) \text{ )} & = \text{Pr(} s_i(\tilde{\alpha}) < \text{ the other bidder's score)} \\ 
%
& = \text{Pr(} s_i(\tilde{\alpha}) < s_j(\alpha_j)\text{ )} \\
%
& = \text{Pr(} s_j^{-1}(s_i(\tilde{\alpha})) < \alpha_j\text{ )} \\
%
& = 1 - F(s_j^{-1}(s_i(\tilde{\alpha})))
\end{align*}

where
\(F(a) = \frac{a - \underline{\alpha}}{ \overline{\alpha} - \underline{\alpha}}\)
is the CDF of the unifrom distribution.

In words --- we're assuming the bidders draw their values \(\alpha_i\)
and \(\alpha_j\) IID from the uniform distribution with cdf \(F(a)\).
Each bidder sees his own \(\alpha\), and knows the bidding functions
\(s_{i}(\cdot)\) and \(s_{j}(\cdot)\), but not the realization of his
opponent's \(\alpha\). The probability that bidder \(i\) will win if he
uses his bidding function \(s_{i}(\alpha_i)\) --- like he's supposed to,
for this to be an equilibirum --- is the probability that
\(s_{i}(\alpha_i)\) is smaller than \(s_{j}(\cdot)\) evaluted at the
(unknown to \(i\)) realization of \(\alpha_j\).

So (plugging in) for a proposed score functions \(s_{i}(\cdot)\) and
\(s_{j}(\cdot)\), the expected utility of participation for bidder \(i\)
can be written:

\[
EU(\tilde{\alpha}, {\gamma_i}, s_{i}(\tilde{\alpha})) = \text{E}[u(\pi(\tilde{\alpha}, {\gamma_i}, s_{i}(\tilde{\alpha}))) \text{ | win}] \times  [1 - F(s_{j}^{-1}(s_{i}(\tilde{\alpha}))) ]
\]

Now the second part -- the expected utility of profits conditional on
winning. The math for this is the same as in the simpler case detailed
in the paper, so I won't replicate it here --- instead I'll present the
formula and explain how it relates to the overall problem.

\[
 \text{E}[u(\pi(\tilde{\alpha}, {\gamma_i}, s_{i}(\tilde{\alpha}))) \text{ | win}]  = \left(1-\exp \left(-{\gamma_i}  \sum_{t=1}^{T} q_{t}^{b}\left(b_{t}^{*}(s_{i}(\tilde{\alpha}))-\tilde{\alpha} c_{t}\right)-\frac{{\gamma_i} \sigma_{t}^{2}}{2}\left(b_{t}^{*}(s_{i}(\tilde{\alpha}))-\tilde{\alpha} c_{t}\right)^{2}\right)\right)
\]

This formula includes a new object: the unit bid
\(b_{t}^{*}(s_{i}(\tilde{\alpha}))\). What is this?

Although only the score determines who wins the auction, in reality,
bidders don't actually submit ``scores'' directly -- instead they submit
a unit bid \(b_t\) for each item \(t\) in the procurement project. The
``score'' for a bidder submitting the vector \(\{b_1,\ldots, b_T\}\) of
bids, is then computed by multiplying by the DOT's estimate of the
quantity of each item that will be needed and summing: \[
\text{score} = \sum\limits_t b_t q_t^e.
\]

But a result from the paper is that we don't have to think too hard
about how to choose the unit bids in equilibrium --- it is sufficient to
solve for an equilibrium \textbf{score} function as we've discussed
before, under the assumption that for any score (and \(\tilde{\alpha}\)
and \({\gamma_i})\) type), bidders will deterministically choose the
unit bids \(b_{t}^{*}(s_{i}(\tilde{\alpha}))\) to maximize
\(\text{E}[u(\pi(\tilde{\alpha}, {\gamma_i}, s_{i}(\tilde{\alpha}))) \text{ | win}]\).

That is: \[
\begin{array}{l}
b_{t}^{*}(s_{i}(\tilde{\alpha})) = \arg\max\limits _{\{ b_t \}} \left[1-\exp \left(-{\gamma_i} \sum_{t=1}^{T} q_{t}^{b}\left(b_{t}-\tilde{\alpha} c_{t}\right)-\frac{{\gamma_i} \sigma_{t}^{2}}{2}\left(b_{t}-\tilde{\alpha} c_{t}\right)^{2}\right)\right] \\ \\ \qquad \qquad \qquad \qquad \qquad \begin{aligned} \text { s.t. } \sum_{t=1}^{T} b_{t} q_{t}^{e}= s_{i}(\tilde{\alpha}) \end{aligned}
\\ \\ \qquad \qquad \qquad \qquad \qquad \begin{aligned} \text { and   }  b_{t} \geq 0 \text{ for each } t\end{aligned}
\end{array}
\]

\textbf{Note:} this really this should be
\(b_{t}^{*}(\tilde{\alpha},{\gamma_i}, s_{i}(\tilde{\alpha}) )\) but I'm
suppressing the other parameters since they're implied.

The optimization program above can be rewritten as a fairly standard
constrained quadratic program:

\[
\begin{array}{l}
b_{t}^{*}(s_{i}(\tilde{\alpha})) = \arg\max\limits _{\{ b_t \}} \left[\sum_{t=1}^{T} \underbrace{q_{t}^{b}\left(b_{t}-\tilde{\alpha} c_{t}\right)}_{\text {Expectation of Profits }} \underbrace{-\frac{{\gamma_i} \sigma_{t}^{2}}{2}\left(b_{t}^{i}-\tilde{\alpha}c_{t}\right)^{2}}_{\text {Variance of Profits }}\right] \\ \\ \qquad \qquad \qquad \qquad \qquad \begin{aligned} \text { s.t. } \sum_{t=1}^{T} b_{t} q_{t}^{e}= s_{i}(\tilde{\alpha}) \end{aligned}
\\ \\ \qquad \qquad \qquad \qquad \qquad \begin{aligned} \text { and   }  b_{t} \geq 0 \text{ for each } t\end{aligned}
\end{array}
\]

We need a numerical solver to solve this in general because there is no
closed form way to know which non-negativity constraints will bind
(e.g.~which bids will be 0 at the optimum.) However, once we solve this,
we know that the solution for any non-zero bid will follow:

\[ b_{t}^{*}(s_{i}(\tilde{\alpha})) = \tilde{\alpha} \cdot c_{t}+ \frac{1}{{\gamma_i}} \cdot \frac{q_{t}^{b}}{\sigma_{t}^{2}}+\frac{q_{t}^{e}}{\sigma_{t}^{2} \sum\limits_{p: b_{p}^{*}>0}\left[\frac{\left(q_{p}^{e}\right)^{2}}{\sigma_{p}^{2}}\right]}\left(s-\sum\limits_{p: b_{p}^{*}>0}\left[\tilde{\alpha} \cdot c_{p}^{o} q_{p}^{e}+\frac{1}{{\gamma_i}}\cdot\frac{q_{p}^{b}}{\sigma_{p}^{2}} q_{p}^{e}\right]\right)
\]

If all of the item bids are above 0 at the optimum, then the summations
in the formula above are over all items \(t = 1, \ldots, T\). Otherwise,
the summations are over the items that have unit bids above 0 at the
optimum.

We can find the partial derivative with respect to \(s\),
\(\frac{\partial b_t(s)}{\partial s}\), from this equation: \[
\frac{\partial b_t(\tilde{s})}{\partial s} = \frac{q_{t}^{e}}{\sigma_{t}^{2} \sum\limits_{p: b_{p}^{*}>0}\left[\frac{\left(q_{p}^{e}\right)^{2}}{\sigma_{p}^{2}}\right]} \quad \text{   if   } \quad b_t(\tilde{s}) > 0
\] and \(\frac{\partial b_t(\tilde{s})}{\partial s} = 0\) if
\(b_t(\tilde{s}) = 0\).

Going back to the expected utility for participating in an auction,
let's go back to a more high level expression:

\[
EU(\tilde{\alpha}, {\gamma_i}, s_{i}(\tilde{\alpha})) = \underbrace{\text{E}[u(\pi(\tilde{\alpha}, {\gamma_i}, s_{i}(\tilde{\alpha}))) \text{ | win}] }_{\text{Value of winning}} \times \underbrace{[1 - F(s_{j}^{-1}(s_{i}(\tilde{\alpha}))) ]}_{\text{Probability of winning}} 
\] and to simplify notation, let's write this as: \[
EU(\tilde{\alpha}, {\gamma_i}, s_{i}(\tilde{\alpha})) \equiv V(s_{i}(\tilde{\alpha})) \times [1 - F(s_{j}^{-1}(s_{i}(\tilde{\alpha}))) ].
\]

We would like to find the function \(s_{i}(\alpha)\) such that bidder
\(i\) will be maximally happy with his bid \(s_{i}({\alpha_i})\) no
matter what draw of \(\alpha_i\) he gets --- meaning that he could not
profit by bidding a different score if his opponent is using
\(s_{j}(\alpha)\) to detrmine \emph{her} bid.

A quick inspection of the
\(EU(\tilde{\alpha}, {\gamma_i}, s_{i}(\tilde{\alpha}))\) function shows
that it is in fact concave in \(s\) (as a value), and so a sufficient
condition for the optimality of \(s(\cdot, \cdot)\) is that the first
order condition of \(EU(\cdot)\) holds:

\[
\frac{\partial EU(\tilde{\alpha}, {\gamma}_i, s_{i}(\tilde{\alpha}))}{\partial \tilde{s}} = 0.
\]

This first order condition is what will define the differential equaiton
that we need to solve in order to find a function \(s(\cdot, \cdot)\)
that will satisfy the conditions to be an ``equilibrium''.

Taking the derivative from our expression above: 
\begin{equation}\label{foc}
\frac{\partial EU(\tilde{\alpha}, {\gamma}_i, s_{i}(\tilde{\alpha}))}{\partial \tilde{s}}  =  
\frac{\partial}{\partial \tilde{s}}V(s_{i}(\tilde{\alpha}))  \times [1 - F(s_{j}^{-1}(s_{i}(\tilde{\alpha}))) ] + V(s_{i}(\tilde{\alpha})) \times \frac{\partial}{\partial \tilde{s}}[1 - F(s_{j}^{-1}(s_{i}(\tilde{\alpha}))) ]
\\
= 0.
\end{equation}

Note that we can compute
\(\frac{\partial}{\partial \tilde{s}}V(s_{i}(\tilde{\alpha}))\) fairly
easily -- this is exactly the same as in the one dimensional type case:

Let's simplify notation one more time and write: \[
CE(s_{i}(\tilde{\alpha})) = \sum_{t=1}^{T} q_{t}^{b}\left(b_{t}^{*}(s_{i}(\tilde{\alpha}))-\tilde{\alpha} c_{t}\right)-\frac{{\gamma_i} \sigma_{t}^{2}}{2}\left(b_{t}^{*}(s_{i}(\tilde{\alpha}))-\tilde{\alpha} c_{t}\right)^{2}
\]

so that

\[
V(s_{i}(\tilde{\alpha}))  = 1 - \exp[-{\gamma_i}CE(s_{i}(\tilde{\alpha}))]
\]

\[
\frac{\partial}{\partial \tilde{s}}V(s_{i}(\tilde{\alpha}))= {\gamma_i} \frac{\partial}{\partial \tilde{s}} CE(s_{i}(\tilde{\alpha}))) \times \exp[-{\gamma_i}CE(s_{i}(\tilde{\alpha}))],
\] where \[
\frac{\partial}{\partial \tilde{s}} CE(s_{i}(\tilde{\alpha})) = \sum_{t=1}^{T} \left[\frac{\partial b_t^{*}(s_{i}(\tilde{\alpha}))}{\partial \tilde{s}} \large\left(q_t^{b} - {\gamma_i} \sigma^{2}(b_t^{*}(s_{i}(\tilde{\alpha},))- \tilde{\alpha} c_t) \large\right)\right].
\]

As noted before, we solve for \(b_t^{*}(s_{i}(\tilde{\alpha}))\) and
\(\frac{\partial b_t^{*}(s_{i}(\tilde{\alpha}))}{\partial s}\)
numerically, and so
\(\frac{\partial}{\partial \tilde{s}} CE( s_{i}(\tilde{\alpha})))\) and
consequently
\(\frac{\partial}{\partial \tilde{s}}V(s_{i}(\tilde{\alpha}))\) are well
defined and computable for any function \(s_{i}(\cdot)\).

Now we can also take the derivative of the second part directly --- but
it will be a function of \(s_{j}^{-1}(\tilde{s})\):

\[
\frac{\partial}{\partial \tilde{s}}[1 - F(s_{j}^{-1}(s_{i}(\tilde{\alpha}))) ] =  [-f(s_{j}^{-1}(s_{i}(\tilde{\alpha})))] \times \frac{\partial s_{j}^{-1}(s_{i}(\tilde{\alpha}))}{\partial \tilde{s}}
\]

\hypertarget{defining-the-inverse-function-and-writing-down-the-ode-sytem}{%
\subsubsection{Defining the inverse function and writing down the ODE
sytem}\label{defining-the-inverse-function-and-writing-down-the-ode-sytem}}

For convenience, we will write the inverse functions \[
\varphi_j(\tilde{s}) \equiv s_{j}^{-1}(\tilde{s}) \text{  and  } \varphi_i(\tilde{s}) \equiv s_{i}^{-1}(\tilde{s}) 
\]

Equilibrium bidding funcitons will alwyas be monotonic (at least for
this class of problems) and so there is a 1:1 relationship between
\(s_{i}(\tilde{\alpha})\) and its inverse \(\varphi_i(\tilde{s})\): \[
\varphi_i(s_{i}(\tilde{\alpha})) = \alpha \text{  and  } s_{i}(\varphi_i(\tilde{s})) = \tilde{s}.
\]

We can thus rewrite the first order condition from Equation (\ref{foc}):
 \[
\frac{\partial EU(\tilde{\alpha}, {\gamma}_i, s_{i}(\tilde{\alpha}))}{\partial \tilde{s}}  =  
\frac{\partial}{\partial \tilde{s}}V(s_{i}(\tilde{\alpha}))  \times [1 - F(\varphi_j(s_{i}(\tilde{\alpha}))) ] + V(s_{i}(\tilde{\alpha})) \times [-f(\varphi_j(s_{i}(\tilde{\alpha})))] \times \frac{\partial \varphi_j(s_{i}(\tilde{\alpha}))}{\partial \tilde{s}}
= 0.
\]

Or equivalently, for each possible \(s_{i}\) and \(s_{j}\):

\[
 \frac{\partial \varphi_j(s_{i})}{\partial \tilde{s}} = \frac{1 - F(\varphi_j(s_{i})) }{f(\varphi_j(s_{i}))} \times \frac{\frac{\partial}{\partial \tilde{s}}V(s_{i})}{V(s_{i})}
\]

and by symmetry:

\[
 \frac{\partial \varphi_i(s_{j})}{\partial \tilde{s}} = \frac{1 - F(\varphi_i(s_{j})) }{f(\varphi_i(s_{j}))} \times \frac{\frac{\partial}{\partial \tilde{s}}V(s_{j})}{V(s_{j})}
\]

This fully defines the system of ODE problem up to a few boundary conditions:

\begin{enumerate}
\def\labelenumi{\arabic{enumi}.}
\tightlist
\item
  The ODEs for \(i\) and \(j\) have to be solved together
\item
  Both boundary scores have to be the same:
\end{enumerate}

\[
\varphi_i(\overline{S}) = \varphi_j(\overline{S}) = \overline{\alpha} \text{ and } \varphi_i(\underline{S}) = \varphi_j(\underline{S}) = \underline{\alpha}
\] or equivalently: \[ 
\overline{S} = s_{i}(\overline{\alpha}) = s_{j}(\overline{\alpha}) \text{  and  } \underline{S} = s_{i}(\underline{\alpha}) = s_{j}(\underline{\alpha})
\] 3. There is a known initial condition that (supposing that
\(\gamma_j < \gamma_i\)):

\[\overline{S} = s_{i}(\overline{\alpha})\]

must be the smallest value of \(s\) s.t.
\(V(s , \overline{\alpha}, \gamma_i) = 0\)

\textbf{Note}: I have to double check the condition for \(\overline{S}\)
above, in (3) is truly correct.

\end{document}
